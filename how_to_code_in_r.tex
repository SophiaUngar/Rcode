% Options for packages loaded elsewhere
\PassOptionsToPackage{unicode}{hyperref}
\PassOptionsToPackage{hyphens}{url}
%
\documentclass[
]{article}
\usepackage{amsmath,amssymb}
\usepackage{iftex}
\ifPDFTeX
  \usepackage[T1]{fontenc}
  \usepackage[utf8]{inputenc}
  \usepackage{textcomp} % provide euro and other symbols
\else % if luatex or xetex
  \usepackage{unicode-math} % this also loads fontspec
  \defaultfontfeatures{Scale=MatchLowercase}
  \defaultfontfeatures[\rmfamily]{Ligatures=TeX,Scale=1}
\fi
\usepackage{lmodern}
\ifPDFTeX\else
  % xetex/luatex font selection
\fi
% Use upquote if available, for straight quotes in verbatim environments
\IfFileExists{upquote.sty}{\usepackage{upquote}}{}
\IfFileExists{microtype.sty}{% use microtype if available
  \usepackage[]{microtype}
  \UseMicrotypeSet[protrusion]{basicmath} % disable protrusion for tt fonts
}{}
\makeatletter
\@ifundefined{KOMAClassName}{% if non-KOMA class
  \IfFileExists{parskip.sty}{%
    \usepackage{parskip}
  }{% else
    \setlength{\parindent}{0pt}
    \setlength{\parskip}{6pt plus 2pt minus 1pt}}
}{% if KOMA class
  \KOMAoptions{parskip=half}}
\makeatother
\usepackage{xcolor}
\usepackage[margin=1in]{geometry}
\usepackage{color}
\usepackage{fancyvrb}
\newcommand{\VerbBar}{|}
\newcommand{\VERB}{\Verb[commandchars=\\\{\}]}
\DefineVerbatimEnvironment{Highlighting}{Verbatim}{commandchars=\\\{\}}
% Add ',fontsize=\small' for more characters per line
\usepackage{framed}
\definecolor{shadecolor}{RGB}{248,248,248}
\newenvironment{Shaded}{\begin{snugshade}}{\end{snugshade}}
\newcommand{\AlertTok}[1]{\textcolor[rgb]{0.94,0.16,0.16}{#1}}
\newcommand{\AnnotationTok}[1]{\textcolor[rgb]{0.56,0.35,0.01}{\textbf{\textit{#1}}}}
\newcommand{\AttributeTok}[1]{\textcolor[rgb]{0.13,0.29,0.53}{#1}}
\newcommand{\BaseNTok}[1]{\textcolor[rgb]{0.00,0.00,0.81}{#1}}
\newcommand{\BuiltInTok}[1]{#1}
\newcommand{\CharTok}[1]{\textcolor[rgb]{0.31,0.60,0.02}{#1}}
\newcommand{\CommentTok}[1]{\textcolor[rgb]{0.56,0.35,0.01}{\textit{#1}}}
\newcommand{\CommentVarTok}[1]{\textcolor[rgb]{0.56,0.35,0.01}{\textbf{\textit{#1}}}}
\newcommand{\ConstantTok}[1]{\textcolor[rgb]{0.56,0.35,0.01}{#1}}
\newcommand{\ControlFlowTok}[1]{\textcolor[rgb]{0.13,0.29,0.53}{\textbf{#1}}}
\newcommand{\DataTypeTok}[1]{\textcolor[rgb]{0.13,0.29,0.53}{#1}}
\newcommand{\DecValTok}[1]{\textcolor[rgb]{0.00,0.00,0.81}{#1}}
\newcommand{\DocumentationTok}[1]{\textcolor[rgb]{0.56,0.35,0.01}{\textbf{\textit{#1}}}}
\newcommand{\ErrorTok}[1]{\textcolor[rgb]{0.64,0.00,0.00}{\textbf{#1}}}
\newcommand{\ExtensionTok}[1]{#1}
\newcommand{\FloatTok}[1]{\textcolor[rgb]{0.00,0.00,0.81}{#1}}
\newcommand{\FunctionTok}[1]{\textcolor[rgb]{0.13,0.29,0.53}{\textbf{#1}}}
\newcommand{\ImportTok}[1]{#1}
\newcommand{\InformationTok}[1]{\textcolor[rgb]{0.56,0.35,0.01}{\textbf{\textit{#1}}}}
\newcommand{\KeywordTok}[1]{\textcolor[rgb]{0.13,0.29,0.53}{\textbf{#1}}}
\newcommand{\NormalTok}[1]{#1}
\newcommand{\OperatorTok}[1]{\textcolor[rgb]{0.81,0.36,0.00}{\textbf{#1}}}
\newcommand{\OtherTok}[1]{\textcolor[rgb]{0.56,0.35,0.01}{#1}}
\newcommand{\PreprocessorTok}[1]{\textcolor[rgb]{0.56,0.35,0.01}{\textit{#1}}}
\newcommand{\RegionMarkerTok}[1]{#1}
\newcommand{\SpecialCharTok}[1]{\textcolor[rgb]{0.81,0.36,0.00}{\textbf{#1}}}
\newcommand{\SpecialStringTok}[1]{\textcolor[rgb]{0.31,0.60,0.02}{#1}}
\newcommand{\StringTok}[1]{\textcolor[rgb]{0.31,0.60,0.02}{#1}}
\newcommand{\VariableTok}[1]{\textcolor[rgb]{0.00,0.00,0.00}{#1}}
\newcommand{\VerbatimStringTok}[1]{\textcolor[rgb]{0.31,0.60,0.02}{#1}}
\newcommand{\WarningTok}[1]{\textcolor[rgb]{0.56,0.35,0.01}{\textbf{\textit{#1}}}}
\usepackage{graphicx}
\makeatletter
\def\maxwidth{\ifdim\Gin@nat@width>\linewidth\linewidth\else\Gin@nat@width\fi}
\def\maxheight{\ifdim\Gin@nat@height>\textheight\textheight\else\Gin@nat@height\fi}
\makeatother
% Scale images if necessary, so that they will not overflow the page
% margins by default, and it is still possible to overwrite the defaults
% using explicit options in \includegraphics[width, height, ...]{}
\setkeys{Gin}{width=\maxwidth,height=\maxheight,keepaspectratio}
% Set default figure placement to htbp
\makeatletter
\def\fps@figure{htbp}
\makeatother
\setlength{\emergencystretch}{3em} % prevent overfull lines
\providecommand{\tightlist}{%
  \setlength{\itemsep}{0pt}\setlength{\parskip}{0pt}}
\setcounter{secnumdepth}{-\maxdimen} % remove section numbering
\ifLuaTeX
  \usepackage{selnolig}  % disable illegal ligatures
\fi
\IfFileExists{bookmark.sty}{\usepackage{bookmark}}{\usepackage{hyperref}}
\IfFileExists{xurl.sty}{\usepackage{xurl}}{} % add URL line breaks if available
\urlstyle{same}
\hypersetup{
  pdftitle={How to Use R},
  pdfauthor={Sophia Ungar},
  hidelinks,
  pdfcreator={LaTeX via pandoc}}

\title{How to Use R}
\author{Sophia Ungar}
\date{2023-02-21}

\begin{document}
\maketitle

I'm not going to go into why use R, except that seems to be primarily
for statistics. I'm not the best person to ask about R vs other stats
languages vs Python. I also want to make this a useful reference guide,
but not a complete teacher.

\hypertarget{other-places-to-learn-things}{%
\subsection{Other places to learn
things}\label{other-places-to-learn-things}}

Suggested by Josh Davidson:
\href{https://r4ds.hadley.nz/data-visualize}{R for Data Science}

\hypertarget{basic-formatting}{%
\subsection{Basic Formatting}\label{basic-formatting}}

RMarkdown combines R code in chunks like this one:

The three ticks and the \{r\} as seen above designate the things inside
as things to be run. You can leave comments that start with a pound
sign, \#, that won't be evaluated.

Outside of the chunks (sorry about the word, not my choice), text is
formatted with markdown. About Markdown:
\url{https://www.markdownguide.org/}

RMarkdown files can be knit to see the final form. Click on the ball of
yarn/ Knit button above this. Or click the drop down arrow to choose
whether to end up with a pdf, HTML, or Word document.

\hypertarget{basic-functions}{%
\subsection{Basic Functions}\label{basic-functions}}

\begin{Shaded}
\begin{Highlighting}[]
\CommentTok{\# Arithmetic}
\DecValTok{2}\SpecialCharTok{+}\DecValTok{3}
\end{Highlighting}
\end{Shaded}

\begin{verbatim}
## [1] 5
\end{verbatim}

\begin{Shaded}
\begin{Highlighting}[]
\DecValTok{2{-}4}
\end{Highlighting}
\end{Shaded}

\begin{verbatim}
## [1] -2
\end{verbatim}

\begin{Shaded}
\begin{Highlighting}[]
\DecValTok{8}\SpecialCharTok{*}\DecValTok{9}
\end{Highlighting}
\end{Shaded}

\begin{verbatim}
## [1] 72
\end{verbatim}

\begin{Shaded}
\begin{Highlighting}[]
\DecValTok{5}\SpecialCharTok{/}\DecValTok{10}
\end{Highlighting}
\end{Shaded}

\begin{verbatim}
## [1] 0.5
\end{verbatim}

\begin{Shaded}
\begin{Highlighting}[]
\DecValTok{4}\SpecialCharTok{**}\DecValTok{10}
\end{Highlighting}
\end{Shaded}

\begin{verbatim}
## [1] 1048576
\end{verbatim}

\begin{Shaded}
\begin{Highlighting}[]
\CommentTok{\# Variable Assignment}
\NormalTok{a }\OtherTok{\textless{}{-}} \DecValTok{2}
\NormalTok{b }\OtherTok{=} \DecValTok{4}

\CommentTok{\# Print}
\NormalTok{b}
\end{Highlighting}
\end{Shaded}

\begin{verbatim}
## [1] 4
\end{verbatim}

\begin{Shaded}
\begin{Highlighting}[]
\DecValTok{5}
\end{Highlighting}
\end{Shaded}

\begin{verbatim}
## [1] 5
\end{verbatim}

\#\#Extra functions we need If you haven't installed the packages first
you will need to run install.packages(``'') in the terminal below or go
to packages in the bottom right.

\begin{Shaded}
\begin{Highlighting}[]
\FunctionTok{library}\NormalTok{(tidyverse)}
\end{Highlighting}
\end{Shaded}

\begin{verbatim}
## Warning: package 'tidyverse' was built under R version 4.3.2
\end{verbatim}

\begin{verbatim}
## Warning: package 'ggplot2' was built under R version 4.3.2
\end{verbatim}

\begin{verbatim}
## Warning: package 'tibble' was built under R version 4.3.2
\end{verbatim}

\begin{verbatim}
## Warning: package 'tidyr' was built under R version 4.3.2
\end{verbatim}

\begin{verbatim}
## Warning: package 'readr' was built under R version 4.3.2
\end{verbatim}

\begin{verbatim}
## Warning: package 'purrr' was built under R version 4.3.2
\end{verbatim}

\begin{verbatim}
## Warning: package 'dplyr' was built under R version 4.3.2
\end{verbatim}

\begin{verbatim}
## Warning: package 'stringr' was built under R version 4.3.2
\end{verbatim}

\begin{verbatim}
## Warning: package 'forcats' was built under R version 4.3.2
\end{verbatim}

\begin{verbatim}
## Warning: package 'lubridate' was built under R version 4.3.2
\end{verbatim}

\begin{verbatim}
## -- Attaching core tidyverse packages ------------------------ tidyverse 2.0.0 --
## v dplyr     1.1.4     v readr     2.1.4
## v forcats   1.0.0     v stringr   1.5.1
## v ggplot2   3.4.4     v tibble    3.2.1
## v lubridate 1.9.3     v tidyr     1.3.0
## v purrr     1.0.2     
## -- Conflicts ------------------------------------------ tidyverse_conflicts() --
## x dplyr::filter() masks stats::filter()
## x dplyr::lag()    masks stats::lag()
## i Use the conflicted package (<http://conflicted.r-lib.org/>) to force all conflicts to become errors
\end{verbatim}

\begin{Shaded}
\begin{Highlighting}[]
\FunctionTok{library}\NormalTok{(ggplot2)}
\end{Highlighting}
\end{Shaded}

\#\#Read in data

\begin{Shaded}
\begin{Highlighting}[]
\CommentTok{\#data \textless{}{-} read.csv(\textless{}filename\textgreater{}.csv) \# getting data from a csv}
\FunctionTok{library}\NormalTok{(palmerpenguins)}
\end{Highlighting}
\end{Shaded}

\begin{verbatim}
## Warning: package 'palmerpenguins' was built under R version 4.3.2
\end{verbatim}

\begin{Shaded}
\begin{Highlighting}[]
\NormalTok{data }\OtherTok{\textless{}{-}}\NormalTok{ penguins}
\end{Highlighting}
\end{Shaded}

We want to pull in data so the computer can look at it. I've commented
out one way to get it from a csv. To use as an example, the
palmerpenguins library has the penguins tibble.

One thing to think about is what kind of structure we want to put the
data in. In R there are tables, data frames, tibbles, and more. Some
functions work on all of them, but some only work on certain types.
TODO: add recommendation on which structure to choose

\hypertarget{clean-up}{%
\subsection{Clean Up}\label{clean-up}}

Get a specific column

\begin{Shaded}
\begin{Highlighting}[]
\NormalTok{flipper }\OtherTok{\textless{}{-}}\NormalTok{ data}\SpecialCharTok{$}\NormalTok{flipper\_length\_mm}
\end{Highlighting}
\end{Shaded}

Isolate or Remove columns

\begin{Shaded}
\begin{Highlighting}[]
\CommentTok{\# same data but only the flipper and bill length columns}
\NormalTok{flipper\_bill }\OtherTok{\textless{}{-}} \FunctionTok{subset}\NormalTok{(data, }\AttributeTok{select=}\FunctionTok{c}\NormalTok{(flipper\_length\_mm, bill\_length\_mm))}

\CommentTok{\# same data but without the island and year columns}
\NormalTok{no\_island\_year }\OtherTok{\textless{}{-}} \FunctionTok{subset}\NormalTok{(data, }\AttributeTok{select=}\SpecialCharTok{{-}}\FunctionTok{c}\NormalTok{(island, year))}
\end{Highlighting}
\end{Shaded}

TODO: add Remove rows with missing data TODO: add find and remove
outliers

\#\#Summarize In R there are many ways to do the same thing. Below is
one way to start looking at the data.

\begin{Shaded}
\begin{Highlighting}[]
\FunctionTok{summary}\NormalTok{(data)}
\end{Highlighting}
\end{Shaded}

\begin{verbatim}
##       species          island    bill_length_mm  bill_depth_mm  
##  Adelie   :152   Biscoe   :168   Min.   :32.10   Min.   :13.10  
##  Chinstrap: 68   Dream    :124   1st Qu.:39.23   1st Qu.:15.60  
##  Gentoo   :124   Torgersen: 52   Median :44.45   Median :17.30  
##                                  Mean   :43.92   Mean   :17.15  
##                                  3rd Qu.:48.50   3rd Qu.:18.70  
##                                  Max.   :59.60   Max.   :21.50  
##                                  NA's   :2       NA's   :2      
##  flipper_length_mm  body_mass_g       sex           year     
##  Min.   :172.0     Min.   :2700   female:165   Min.   :2007  
##  1st Qu.:190.0     1st Qu.:3550   male  :168   1st Qu.:2007  
##  Median :197.0     Median :4050   NA's  : 11   Median :2008  
##  Mean   :200.9     Mean   :4202                Mean   :2008  
##  3rd Qu.:213.0     3rd Qu.:4750                3rd Qu.:2009  
##  Max.   :231.0     Max.   :6300                Max.   :2009  
##  NA's   :2         NA's   :2
\end{verbatim}

\begin{Shaded}
\begin{Highlighting}[]
\FunctionTok{head}\NormalTok{(data) }\CommentTok{\# see the first rows. 6 rows by default}
\end{Highlighting}
\end{Shaded}

\begin{verbatim}
## # A tibble: 6 x 8
##   species island    bill_length_mm bill_depth_mm flipper_length_mm body_mass_g
##   <fct>   <fct>              <dbl>         <dbl>             <int>       <int>
## 1 Adelie  Torgersen           39.1          18.7               181        3750
## 2 Adelie  Torgersen           39.5          17.4               186        3800
## 3 Adelie  Torgersen           40.3          18                 195        3250
## 4 Adelie  Torgersen           NA            NA                  NA          NA
## 5 Adelie  Torgersen           36.7          19.3               193        3450
## 6 Adelie  Torgersen           39.3          20.6               190        3650
## # i 2 more variables: sex <fct>, year <int>
\end{verbatim}

\#\#Basic Stats Functions

\begin{Shaded}
\begin{Highlighting}[]
\FunctionTok{mean}\NormalTok{( data}\SpecialCharTok{$}\NormalTok{flipper\_length\_mm, }\AttributeTok{na.rm=}\ConstantTok{TRUE}\NormalTok{) }\CommentTok{\# mean, ignoring missing values}
\end{Highlighting}
\end{Shaded}

\begin{verbatim}
## [1] 200.9152
\end{verbatim}

\begin{Shaded}
\begin{Highlighting}[]
\FunctionTok{sd}\NormalTok{( flipper, }\AttributeTok{na.rm=}\ConstantTok{TRUE}\NormalTok{) }\CommentTok{\# standard deviation, ignoring missing values}
\end{Highlighting}
\end{Shaded}

\begin{verbatim}
## [1] 14.06171
\end{verbatim}

TODO: add confidence intervals

t test

ANOVA

\#\#Visualize

This example shows flipper length by bill length. The color is the sex
of the bird.

\begin{Shaded}
\begin{Highlighting}[]
\FunctionTok{ggplot}\NormalTok{(data, }\FunctionTok{aes}\NormalTok{(flipper\_length\_mm, bill\_length\_mm, }\AttributeTok{color=}\NormalTok{sex)) }\SpecialCharTok{+}
    \FunctionTok{geom\_point}\NormalTok{()}
\end{Highlighting}
\end{Shaded}

\begin{verbatim}
## Warning: Removed 2 rows containing missing values (`geom_point()`).
\end{verbatim}

\includegraphics{how_to_code_in_r_files/figure-latex/unnamed-chunk-10-1.pdf}

\#\#Simulate

\end{document}
